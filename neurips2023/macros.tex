% For theorems and such
\usepackage{amsmath}
\usepackage{amssymb}
\usepackage{mathtools}
\usepackage{amsthm}
\usepackage{nicefrac}
\usepackage{multirow}
\usepackage{tikz-cd} %for tikzcd
\usepackage{subcaption}
\usepackage{multicol,caption}
\newenvironment{Figure}
  {\par\medskip\noindent\minipage{\linewidth}}
  {\endminipage\par\medskip}
\usepackage{wrapfig}

\usepackage[flushleft]{threeparttable} % http://ctan.org/pkg/threeparttable
\usepackage{colortbl}
\definecolor{bgcolor}{rgb}{0.8,1,1}
\definecolor{bgcolor2}{rgb}{0.8,1,0.8}
\definecolor{niceblue}{rgb}{0.0,0.19,0.56}

\usepackage{hyperref}
\hypersetup{colorlinks,linkcolor={blue},citecolor={niceblue},urlcolor={blue}}

% \usepackage[dvipsnames]{xcolor}
\usepackage{pifont}
\definecolor{PineGreen}{RGB}{0,110,51}
\definecolor{BrickRed}{RGB}{143,20,2}
\newcommand{\cmark}{{\color{PineGreen}\ding{51}}}%
\newcommand{\xmark}{{\color{BrickRed}\ding{55}}}%

\usepackage{thmtools}

\newcommand{\cD}{{\cal D}}
\newcommand{\Exp}{\mathbb{E}}
\newcommand{\Prob}[1]{\mathbb{P} \left[ #1\right]}
% \usepackage[colorinlistoftodos,bordercolor=orange,backgroundcolor=orange!20,linecolor=orange,textsize=scriptsize]{todonotes}
% \newcommand{\TODO}[1]{\todo[inline,linecolor=green,backgroundcolor=green!25,bordercolor=green,caption={}]{#1}}
% \newcommand{\nicolas}[1]{\todo[inline]{\textbf{Nicolas: }#1}}
% \newcommand{\eduard}[1]{\todo[inline]{\textbf{Eduard: }#1}}
% \newcommand{\sayantan}[1]{\todo[inline]{\textbf{Sayantan: }#1}}


 \usepackage[disable,textsize=tiny]{todonotes}
% \usepackage[textsize=tiny]{todonotes}

%macros for comments
\providecommand{\mycomment}[3]{\todo[inline,caption={},color=#3!20]{\textbf{#1: }#2}}%
\providecommand{\inlinecomment}[3]{{\color{#1}#2: #3}}
\newcommand\commenter[2]%
{%
  \expandafter\newcommand\csname i#1\endcsname[1]{\inlinecomment{#2}{#1}{##1}}
  \expandafter\newcommand\csname #1\endcsname[1]{\mycomment{#1}{##1}{#2}}
}

%%%%%%%%%%%%%%%%%%%%%%%%%%%%%%%%%%%%%%%
% To remove
%%%%%%%%%%%%%%%%%%%%%%%%%%%%%%%%%%%%%%%
\commenter{cqmo}{red}
\commenter{jhjd}{blue}
\commenter{hnb}{orange}
\commenter{Nicolas}{yellow}
\commenter{Eduard}{green}
\commenter{Sayantan}{white}

%%%% Shaded theorems
\definecolor{shadecolor}{gray}{0.9}
\declaretheoremstyle[
headfont=\normalfont\bfseries,
notefont=\mdseries, notebraces={(}{)},
bodyfont=\normalfont,
postheadspace=0.5em,
spaceabove=1pt,
mdframed={
  skipabove=8pt,
  skipbelow=8pt,
  hidealllines=true,
  backgroundcolor={shadecolor},
  innerleftmargin=4pt,
  innerrightmargin=4pt}
]{shaded}


\declaretheorem[style=shaded,within=section]{definition}
\declaretheorem[style=shaded,sibling=definition]{theorem}
\declaretheorem[style=shaded,sibling=definition]{proposition}
\declaretheorem[style=shaded,sibling=definition]{assumption}
\declaretheorem[style=shaded,sibling=definition]{corollary}
\declaretheorem[style=shaded,sibling=definition]{conjecture}
\declaretheorem[style=shaded,sibling=definition]{lemma}
\declaretheorem[style=shaded,sibling=definition]{remark}
\declaretheorem[style=shaded,sibling=definition]{example}
% \declaretheorem[style=shaded,sibling=definition]{algorithm}


% if you use cleveref..
\usepackage[capitalize,noabbrev]{cleveref}


% Attempt to make hyperref and algorithmic work together better:
%\newcommand{\theHalgorithm}{\arabic{algorithm}}

\usepackage{xspace}

\newcommand{\algname}[1]{{\sf  #1}\xspace}
\newcommand{\algnamex}[1]{{\sf #1}\xspace}

\newcommand{\R}{\mathbb{R}}
\newcommand{\N}{\mathbb{N}}
\newcommand{\E}{\mathbb{E}}
\newcommand{\cO}{{\cal O}}

\newcommand{\hx}{\hat{x}}
\newcommand{\la}{\left\langle}
\newcommand{\g}{\gamma}
\newcommand{\ra}{\right\rangle}
\newcommand{\vr}{\varphi}
\newcommand{\vep}{\varepsilon}
\newcommand{\lam}{\lambda}
\newcommand{\om}{\omega}
\newcommand{\eqdef}{:=}

\DeclarePairedDelimiter\ceil{\lceil}{\rceil}